\documentclass[12pt]{article}
\usepackage{fancyhdr}
\usepackage{lipsum}
\usepackage{amsmath}
\usepackage{amssymb}
\usepackage{geometry}
\usepackage{listings}
\usepackage{inconsolata}

\lstset{
    basicstyle=\ttfamily, % Use the monospaced font
    numbers=left,         % Display line numbers on the left
    numberstyle=\tiny,    % Style of the line numbers
    stepnumber=1,         % Step between two line-numbers
    numbersep=5pt,        % Distance between the line numbers and the code
    frame=single,         % Adds a frame around the code
    breaklines=true,      % Wrap long lines
    postbreak=\mbox{$\hookrightarrow$\space}, % Adds a symbol at the end of wrapped lines
}

\pagestyle{fancy}
\fancyhf{}
\fancyhead[R]{\thepage}
\fancyhead[L]{Innopolis University (Timolai Andrievich, Ammar Meslmani, Laith Nayal)}
\renewcommand{\headrulewidth}{0pt}

\geometry{margin=0.5in, top=40pt, bottom=1in}
\setlength{\headheight}{40pt}


\begin{document}

\tableofcontents

% \section{Templates}
% \emph{This section is an example for adding sections to TRD, needs to be removed before printing.}
%
% \subsection{C++ template}
% A simple C++ template with functions for inputting/outputting vectors.
\lstinputlisting[language=c++]{./templates/cpp-template/code.cpp}

%
% \subsection{Python template}
% A simple python template.
\lstinputlisting[language=Python]{./templates/py-template/code.py}


\section{Algorithms}

\subsection{FFT}
Fast Fourier Transform is an algorithm that, given a polynomial $P$ of degree $n = 2^k - 1$, returns the evaluation of that polynomial at points $w^0, w^1, ..., w^{n - 2}, w^{n - 1} = \mathcal{F}(P)$. That representation allows for fast evaluation of some operations on polynomials, for example: $\mathcal{F}(P \ast Q) = \mathcal{F}(P) \odot \mathcal{F}(Q)$. It is also possible to evaluate FFT in some rings $\mathbb{Z}_p$, where $p$ is some prime. In particular, $31$ is the $2^{23}$-th root of unity in the ring $\mathbb{Z}/\mathbb{Z}{998244353}$.

\lstinputlisting[language=c++]{./algorithms/fast-fourier-transform/code.cpp}


\subsection{Z-function}
Z-function of string $S$, index $i$ is the length of the longest prefix of $S$, such that it coincides with prefix of slice $S[i:]$. It can be used for searching for a string $T$ in the string $S$, by calculating the Z-function of $T\#S$, where $\#$ is some character not present in $S$ or $T$. Z-function can be calculated in $O(S + T)$ time, which gives an advantage over the naive method.

\lstinputlisting[language=c++]{./algorithms/z-function/code.cpp}


\subsection{Fast Exponentiation}
Fast exponentiation $x^p$ in some ring $Z_{\text{mod}}$ in $O(\log p)$.

\lstinputlisting[language=c++]{./algorithms/fastpow/code.cpp}





\section{Data Structures}

\subsection{Fenwick Tree}
Updating the $i$-th element of the Fenwick tree updates all elements with indicies $j \geq i$. As such, it can be used to perform min/max queries on the suffix of an array. It also can be used for range updates if the operation is reversible (i.e. $\forall x\left[\exists -x: f(x, -x) = 0_f\right]$

\lstinputlisting[language=c++]{./data-structures/fenwick-tree/code.cpp}


\subsection{Segment Tree}
Segment tree is a data structure that allows for range queries and pointwise updates. To build a segment tree on group $(G, \odot)$, the following conditions must hold:
\begin{enumerate}
  \item $\exists 0_{\odot}: \forall x \in G \left[ x \odot 0_{\odot} = x \right]$ (The tree still can be built, but without padding, which is slightly less convenient.)
  \item $\forall x \in G, y \in G \left[ x \odot y = y \odot x \right]$ (The operation must be associative, otherwise the result of the query will be basically random.)
\end{enumerate}
The time complexity of the segment tree is $O(n)$ to build, $O(n \log n)$ to update and query.
\\ \\
\lstinputlisting[language=c++]{./data-structures/segment-tree/code.cpp}



\section{Math}

\subsection{Correct Bracket Sequences}
The set of correct bracket sequences $\mathbb{S}$ can be defined in a following way:
\begin{enumerate}
  \item $\epsilon \in \mathbb{S}$
  \item $\forall (s_1, s_2) \in \mathbb{S}^2\left["(" \frown s_1 \frown ")" \frown s_2 \in \mathbb{S}\right]$
\end{enumerate}
From this definition, the following recurrent formula for the number of correct bracket sequences of length $n$ can be derived:
\begin{align*}
  C_0 &= 1 \\
  C_n &= \sum\limits_{i=0}^{n - 1} C_i C_{n - 1 - i}
\end{align*}
\\ \\
The analytical formula is:
\begin{equation*}
  C_n = \binom{2n}{n} \frac{1}{n + 1}
\end{equation*}



\subsection{Polynomial Shift}
Given a polynomial $P(x)$ of degree $n$, the task is to calculate coefficients of $Q(x) = P(x + a)$. To do that, calculate: 
\begin{align*}
U_k &= P_{n-k} (n - k)! \\
V_k &= \frac{a^k}{k!} \\
G &= (U \ast V) \\
Q_k &= \frac{G_k}{k!}
\end{align*}
\lstinputlisting[language=c++]{./math/polynomial-shift/code.cpp}


\end{document}
